\documentclass[10pt,landscape]{article}
\usepackage[T1]{fontenc}
\usepackage{multicol}
\usepackage{calc}
\usepackage{ifthen}
\usepackage[landscape]{geometry}
\usepackage{amsmath,amsthm,amsfonts,amssymb}
\usepackage{color,graphicx,overpic}
\usepackage{hyperref}
\usepackage[os=mac, mackeys=symbols]{menukeys}


\pdfinfo{
  /Title (example.pdf)
  /Creator (TeX)
  /Producer (pdfTeX 1.40.0)
  /Author (Seamus)
  /Subject (Example)
  /Keywords (pdflatex, latex,pdftex,tex)}

% This sets page margins to .5 inch if using letter paper, and to 1cm
% if using A4 paper. (This probably isn't strictly necessary.)
% If using another size paper, use default 1cm margins.
\ifthenelse{\lengthtest { \paperwidth = 11in}}
    { \geometry{top=.25in,left=.25in,right=.25in,bottom=.25in} }
    {\ifthenelse{ \lengthtest{ \paperwidth = 297mm}}
        {\geometry{top=1cm,left=1cm,right=1cm,bottom=1cm} }
        {\geometry{top=1cm,left=1cm,right=1cm,bottom=1cm} }
    }

% Turn off header and footer
\pagestyle{empty}

% Redefine section commands to use less space
\makeatletter
\renewcommand{\section}{\@startsection{section}{1}{0mm}%
                                {-1ex plus -.5ex minus -.2ex}%
                                {0.5ex plus .2ex}%x
                                {\normalfont\large\bfseries}}
\renewcommand{\subsection}{\@startsection{subsection}{2}{0mm}%
                                {-1explus -.5ex minus -.2ex}%
                                {0.5ex plus .2ex}%
                                {\normalfont\normalsize\bfseries}}
\renewcommand{\subsubsection}{\@startsection{subsubsection}{3}{0mm}%
                                {-1ex plus -.5ex minus -.2ex}%
                                {1ex plus .2ex}%
                                {\normalfont\small\bfseries}}
\makeatother

% Define BibTeX command
\def\BibTeX{{\rm B\kern-.05em{\sc i\kern-.025em b}\kern-.08em
    T\kern-.1667em\lower.7ex\hbox{E}\kern-.125emX}}

% Don't print section numbers
\setcounter{secnumdepth}{0}


\setlength{\parindent}{0pt}
\setlength{\parskip}{0pt plus 0.5ex}

%My Environments
\newtheorem{example}[section]{Example}
% -----------------------------------------------------------------------

\begin{document}
\raggedright
\footnotesize
\begin{multicols}{3}

% multicol parameters
% These lengths are set only within the two main columns
%\setlength{\columnseprule}{0.25pt}
\setlength{\premulticols}{1pt}
\setlength{\postmulticols}{1pt}
\setlength{\multicolsep}{1pt}
\setlength{\columnsep}{2pt}

\begin{center}
     \large{\underline{Sublime Text Cheatsheet (Mac)}} \\
     \small{Version 1.5, updated \today}
\end{center}

\vspace*{-\baselineskip}
\section{General}
\keys{\cmd+\shift+P} Command Palette \\
\keys{\cmd+P} \textbf{Goto Anything} \\
\keys{\cmd+R} \textbf{Goto Symbol} \\
\keys{\cmd+N} New File \\
\keys{\cmd+S} Save File \\
\keys{\cmd+\shift+S} Save File As \\
\keys{\cmd+\Alt+S} Save All \\
\keys{\cmd+O} Open File \\
\keys{\cmd+W} Close File \\
\keys{\cmd+\shift+T} Reopen Closed File \\
\keys{\cmd+\shift+N} New Window \\
\keys{\cmd+\shift+W} Close Window \\
% \keys{\cmd+K}, \keys{\cmd+B} Show Sidebar \\
\keys{\ctrl+S} \textbf{Show Sidebar} \textsc{remap} \\
\keys{\cmd+\ctrl+M} \textbf{Toggle Minimap} \textsc{remap} \\
\keys{\ctrl+`} Show Console \\
\keys{\cmd+\ctrl+F} Enter Full Screen \\
\keys{\cmd+\ctrl+\shift+F} \textbf{Enter Distraction-Free Mode} \\
\keys{\cmd+\ctrl+P} \textbf{Switch Projects}

\section{Moving/editing}
\keys{\cmd+Z} Undo \\
\keys{\cmd+\shift+Z} Redo \\
\keys{\cmd+Y} Repeat \\
\keys{\cmd+C} Copy \\
\keys{\ctrl+Y} Yank (inserts the text that's on top of the kill ring)\\
\keys{\cmd+X} Cut \\
\keys{\cmd+V} Paste \\
\keys{\cmd+\shift+V} Paste and Indent \\
\keys{\cmd+]} Indent \\
\keys{\cmd+[} Unindent \\
\keys{\cmd+\ctrl+\arrowkeyup}/\keys{\arrowkeydown} Swap Line Up/Down \\
\keys{\cmd+\shift+D} \textbf{Duplicate Line} \\
\keys{\ctrl+\shift+K} Delete (kill) Line \\
\keys{\cmd+\shift+K} Delete (kill) Line \\
\keys{\cmd+J} \textbf{Join Lines} \\
\keys{\cmd+/} Toggle Comment \\
\keys{\cmd+\Alt+/} Toggle Block Comment \\
\keys{\cmd+\shift+\return} Insert Line Before \\
\keys{\cmd+\return} Insert Line After \\
\keys{\cmd+\backspace} Delete to Beginning of Line \\
\keys{\ctrl+K} Delete to End of Line \\
\keys{\ctrl+A} Move to Beginning of Line \\
\keys{\ctrl+E} Move to End of Line \\
\keys{\ctrl+P}(revious) Move Up \\
\keys{\ctrl+F}(orward) Move Right \\
\keys{\ctrl+N}(ext) Move Down \\
\keys{\ctrl+B}(ack) Move Left \\
\keys{\cmd+\ctrl+K} Swap Line Up \textsc{remap} \\
\keys{\cmd+\ctrl+J} Swap Line Down \textsc{remap} \\
\keys{\ctrl+T} Transpose \\
\keys{\cmd+\Alt+.} \textbf{Close Tag} \\
\keys{\ctrl+\shift+W} Wrap Selection with Tag \\
\keys{\cmd+K}, \keys{\cmd+U} Convert Selected Text to Uppercase \\
\keys{\cmd+K}, \keys{\cmd+L} Convert Selected Text to Lowercase \\
\keys{\cmd+K}, \keys{\cmd+T} Convert Selected Text to Titlecase \\
\keys{\cmd+\Alt+Q} Wrap Paragraph at Ruler \\
\keys{F5} \textbf{Sort Lines} \\
\keys{\ctrl+F5} Sort Lines (Case Sensitive) \\
\keys{F8} \textbf{Shuffle Lines} \textsc{remap} \\
\keys{\ctrl+G} Goto Line \\
\keys{\ctrl+M} \textbf{Jump to Matching Bracket} \\
\keys{\ctrl+L} Scroll to Selection (centers the screen to cursor) \\
\keys{\ctrl+\Alt+\arrowkeyup}/\keys{\arrowkeydown} Scroll Line Up/Down \\
\keys{\ctrl+Q} Record Macro \\
\keys{\ctrl+\shift+Q} Playback Macro

\section{Selection}
\keys{\cmd+U} Undo \\
\keys{\cmd+\shift+U} Soft Redo \\
\keys{\cmd+\shift+L} \textbf{Split Selection Into Lines} (inserts multiple cursors) \\
\keys{\ctrl+\shift+\arrowkeyup}/\keys{\arrowkeydown} Add Previous/Next Line (inserts multiple cursors) \\
\keys{\esc} Single Selection (when there are multiple selections) \\
\keys{\cmd+click} \textbf{Create Cursors Anywhere} \\
\keys{\cmd+A} Select All \\
\keys{\cmd+D} \textbf{Expand Selection to Word (Then Next Match)} \\
\keys{\cmd+L} Expand Selection to Line \\
\keys{\cmd+\shift+A} Expand Selection to Tag (HTML/XML) \\
\keys{\cmd+\shift+\space} \textbf{Expand Selection to Scope} \\
\keys{\ctrl+\shift+M} Expand Selection to Brackets \\
\keys{\cmd+\shift+J} Expand Selection to Indentation \\

\section{Search/replace}
\keys{\cmd+F} Find \\
\keys{\cmd+G} Find Next \\
\keys{\cmd+\shift+G} Find Previous \\
\keys{\cmd+I} Incremental Find \\
\keys{\cmd+\Alt+F} Replace \\
\keys{\cmd+\Alt+E} Replace Next \\
\keys{\cmd+\Alt+G} Quick Find (searches for the word under the cursor) \\
\keys{\cmd+\ctrl+G} \textbf{Quick Find All} (selects all occurences of the word under the cursor) \\
\keys{\cmd+E} Use Selection for Find \\
\keys{\cmd+\shift+E} Use Selection for Replace \\
\keys{\cmd+\shift+F} Find in Files (all open files) \\
\keys{F4} Next Result (file search results) \\
\keys{\shift+F4} Previous Result (file search results)

\section{Folding}
\keys{\cmd+\Alt+[} Fold (selection) \\
\keys{\cmd+\Alt+]} Unfold \\
\keys{\cmd+K}, \keys{\cmd+J} Unfold All \\
\keys{\cmd+K}, \keys{\cmd+1} Fold All \\
\keys{\cmd+K}, \keys{\cmd+1-9} Fold Level 1-9 \\
\keys{\cmd+K}, \keys{\cmd+T} Fold Tag Attributes (HTML/XML)

\section{Bookmarking}
\keys{\cmd+K}, \keys{\cmd+\space} Set Mark \\
\keys{\cmd+K}, \keys{\cmd+A} Select to Mark \\
\keys{\cmd+K}, \keys{\cmd+W} Delete to Mark \\
\keys{\cmd+K}, \keys{\cmd+X} Swap with Mark \\
\keys{\cmd+K}, \keys{\cmd+G} Clear Mark \\
\keys{\cmd+F2} Toggle Bookmark \\
\keys{F2} Next Bookmark \\
\keys{\shift+F2} Previous Bookmark \\
\keys{\cmd+\shift+F2} Clear Bookmarks

\section{Splits/Tabs}
\keys{\cmd+\Alt+1} Single Column \\
\keys{\cmd+\Alt+2} Two Columns \\
\keys{\cmd+\Alt+5} Grid \\
\keys{\ctrl+1-4} Focus Group \\
\keys{\ctrl+\shift+1-4} Move File to Group \\
\keys{\cmd+1-4} Select Tab \\

\section{Miscellaneous}
\keys{F6} \textbf{Toggle Spell Check} \\
\keys{\ctrl+F6} Next Misspelling \\
\keys{\ctrl+\shift+F6} Previous Misspelling \\
\keys{\cmd+B} Build \\
\keys{\ctrl+\space} Show Completions \\
\keys{\cmd+\Alt+T} Special Characters

\subsection{\LaTeX{}ing package}
\keys{\cmd+B} Save And Build \\
\keys{\cmd+\shift+B} Save, Build, Switch to pdf \\
\keys{\ctrl+C} Kill build \\
\keys{\cmd+L}, \keys{\cmd+J} Forward search from tex to pdf \\
\keys{\cmd+L}, \keys{\cmd+O} Open pdf \\
% \keys{\cmd+L}, \keys{T}, \keys{S} Toggle pdf syncing after build \\
% \keys{\cmd+L}, \keys{T}, \keys{?} Status of toggles \\
\keys{\cmd+L}, \keys{\backspace} Remove temp files \textsc{remap} \\
\keys{\cmd+R} Goto sections and labels \\
\keys{\cmd+L}, \keys{\cmd+S} \textbf{Show Snippets} \\
\keys{\cmd+L}, \keys{\cmd+C} Turn word into command \\
\keys{\cmd+L}, \keys{\cmd+E} Turn word into environment \\
begin,\keys{\tab} New environment \\
% \keys{\cmd+L}, \keys{\cmd+C} Selection to command \\
\keys{\cmd+L}, \keys{\cmd+N} Selection to environment \\
\keys{\cmd+L}, \keys{.} Close environment \\
\keys{\cmd+L}, \keys{\cmd+W}, \keys{E} Emphasis/italics \\
\keys{\cmd+L}, \keys{\cmd+W}, \keys{B} Bold \\
\keys{\cmd+L}, \keys{\cmd+W}, \keys{U} Underlined \\
\keys{\cmd+L}, \keys{\cmd+W}, \keys{T} Typewriter  \textsc{remap} \\
\keys{\shift+\return} \textbf{Auto newline (table or item)} \\
\keys{\cmd+L}, \keys{\cmd+L} Auto-fill command \\
\keys{\cmd+L}, \keys{\cmd+T} Toggle preferences \\

\subsection{Markdown Preview package}
\keys{\Alt+M} Preview Markdown in Browser \textsc{remap} \\

\subsection{AdvancedNewFile package}
\keys{\cmd+\Alt+N} New ./path/to/file from file/project directory \\
% \keys{\cmd+\shift+\Alt+N} New file from file/project directory (adds an \texttt{\_\_init\_\_.py} file to each directory created)

\subsection{Alignment package}
\keys{\cmd+\shift+A} Align by Delimiter \textsc{remap} \\

\subsection{Emmet package}
\keys{\ctrl+X} Expand abbreviation (e.g. \texttt{ul>li.listitem*6>a}) \textsc{remap} \\
\keys{P}, \keys{2}, \keys{0}, \keys{\tab} Expands to \texttt{padding: 20px} \\
\keys{F}, \keys{Z}, \keys{\tab} Expands to \texttt{font-size:} \\
\keys{\shift+\cmd+Y} \textbf{Evaluate Math Expression}

\subsection{Git package}

\subsection{Preference Helper package}
\keys{\cmd+\shift+,} Align by Delimiter \textsc{remap} \\



super+shift+, on OSX or ctrl+shift+, on Windows/Linux shows a quick panel with all available sublime-settings files, you can filter and open one from here.


super+alt+up on OSX or alt+o on Windows/Linux toggles between the user and default sublime-settings file

super+shift+; on OSX or ctrl+shift+; on Windows/Linux completes the value of the current key with the initial value of the default sublime-setting.

\end{multicols}

\end{document}
